% Created 2018-06-21 Thu 12:21
% Intended LaTeX compiler: pdflatex
\documentclass[11pt]{article}
\usepackage[utf8]{inputenc}
\usepackage[T1]{fontenc}
\usepackage{graphicx}
\usepackage{grffile}
\usepackage{longtable}
\usepackage{wrapfig}
\usepackage{rotating}
\usepackage[normalem]{ulem}
\usepackage{amsmath}
\usepackage{textcomp}
\usepackage{amssymb}
\usepackage{capt-of}
\usepackage{hyperref}
\author{Vishakh Kumar}
\date{\today}
\title{}
\hypersetup{
 pdfauthor={Vishakh Kumar},
 pdftitle={},
 pdfkeywords={},
 pdfsubject={},
 pdfcreator={Emacs 25.3.1 (Org mode 9.1.13)}, 
 pdflang={English}}
\begin{document}

\tableofcontents




\section{Obtaining html from source}
\label{sec:orgf2cf69e}


\begin{itemize}
\item Why are we using selenium?
The website we're trying to parse has dynamic content (it's live data that refreshes every 15 minutes), which requires us to open a webbrowser and actually render the page. Selenium allows us to open a webbrowser and actually render said dynamic content. Without selenium, we'd be stuck with a (rather prettily designed) html page without information.

\item Why does this open a web browser?
Selenium requires an actual webbrowser to do stuff. I'll change to a better solution soon but it's Firefox for now. Phantomjs seems to be a good option.
\end{itemize}

\begin{verbatim}
def html_source(path):
    # Import a web browser
    from selenium import webdriver

    # Open the webbrowser with a page
    driver = webdriver.Firefox()
    driver.get(path)

    # Obtain html source from window
    html = driver.page_source
    return html
\end{verbatim}
\captionof{figure}{\label{orgc7a2de3}
Function to take html content from a website}

\begin{itemize}
\item Which websites are we taking information from?
Since this particular script is focused on Nasdaq Dubai, we'll be focusing on their market data, available \href{http://www.nasdaqdubai.com/marketdata/\#/equities}{here}. They have three main types of information - equities and funds, securities, and bonds.
\end{itemize}

\begin{verbatim}
html_equities = html_source("http://www.nasdaqdubai.com/marketdata/#/equities")
\end{verbatim}


\section{Parsing html}
\label{sec:org25f0c6f}

\begin{verbatim}
# <table class="table table-striped data-header">
\end{verbatim}

\begin{verbatim}
def html_parse(html):
    import bs4
    import pandas as pd


    soup = bs4.BeautifulSoup(html,"html5lib")
    tables = soup.find_all('table')
    for table in tables:
        for tbody in table.select('tbody'):
            print('------')
            for tr in tbody.select('tr'):
                print('++++++++++++++++++++++++++')
                #print(tr)

                # Company Information
                full_name = tr.select('td[class="full_name ng-binding"]')[0].getText()
                code = tr.select('td[class="code ng-binding"]')[0].getText()

                # Yesterday's price and highs and lows
                currency = tr.select('td[class="curr ng-binding"]')[0].getText()
                previous_close = float(tr.select('td[class="previous_close ng-binding"]')[0].getText())
                open_ = float(tr.select('td[class="open ng-binding"]')[0].getText())
                high = float(tr.select('td[class="high ng-binding"]')[0].getText())
                low = float(tr.select('td[class="low ng-binding"]')[0].getText())

                # Bids and Offers
                bid_qty = (tr.select('td[class="bid_qty ng-binding"]')[0].getText())
                bid = (tr.select('td[class="bid ng-binding"]')[0].getText())
                offer = (tr.select('td[class="offer ng-binding"]')[0].getText())
                offer_qty = (tr.select('td[class="offer_qty ng-binding"]')[0].getText())

                # Last traded price
                last_traded_price = float(tr.select('td[class="last_traded_price ng-binding"]')[0].getText())
                percentage_change = float(tr.select('td[class="percentage_change ng-binding"]')[0].getText())


                dictionary = {"Company Name":full_name,
                              "Code":code,
                              "Currency":currency,
                              "Previous Close":previous_close,
                              "Open":open_,
                              "High":high,
                              "Low":low,
                              "Bid Quantity":bid_qty,
                              "Bid":bid,
                              "Offer":offer,
                              "Offer Quantity":offer_qty,
                              "Last Traded Price":last_traded_price,
                              "Percentage Change":percentage_change}
                print(dictionary)
                print('++++++++++++++++++++++++++')
            print('------')
\end{verbatim}
\captionof{figure}{\label{org406c4ab}
Function to parse the html content we get and return data as a pandas\(_{\text{frame}}\)}

\begin{verbatim}
stuff = html_parse(html)
\end{verbatim}
\end{document}